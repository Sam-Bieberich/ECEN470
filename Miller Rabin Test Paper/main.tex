%--------------------
% Packages
% -------------------
\documentclass[11pt,a4paper]{article}
\usepackage[utf8x]{inputenc}
\usepackage[T1]{fontenc}
%\usepackage{gentium}
\usepackage{mathptmx} % Use Times Font


\usepackage[pdftex]{graphicx} % Required for including pictures
\usepackage[english]{babel}
\usepackage[pdftex,linkcolor=black,pdfborder={0 0 0}]{hyperref} % Format links for pdf
\usepackage{calc} % To reset the counter in the document after title page
\usepackage{enumitem} % Includes lists

\frenchspacing % No double spacing between sentences
\linespread{1.2} % Set linespace
\usepackage[a4paper, lmargin=0.1666\paperwidth, rmargin=0.1666\paperwidth, tmargin=0.1111\paperheight, bmargin=0.1111\paperheight]{geometry} %margins
%\usepackage{parskip}

\usepackage[all]{nowidow} % Tries to remove widows
\usepackage[protrusion=true,expansion=true]{microtype} % Improves typography, load after fontpackage is selected

\usepackage{lipsum} % Used for inserting dummy 'Lorem ipsum' text into the template


%-----------------------
% Set pdf information and add title, fill in the fields
%-----------------------
\hypersetup{ 	
pdfsubject = {},
pdftitle = {},
pdfauthor = {}
}

\usepackage{fancyhdr}

% Clear default header and footer
\fancyhf{}

% Your name goes in the header
\lhead{Samuel Bieberich - MATH 470 Honors Project 2, Miller-Rabin Test}

% Set the page style to use the fancyhdr settings
\pagestyle{fancy}

\usepackage{listings}
\usepackage{xcolor}

\definecolor{mygreen}{rgb}{0.0, 0.5, 0.0}

\lstdefinestyle{mystyle}{
    language=Python,
    basicstyle=\ttfamily,
    keywordstyle=\color{blue},
    commentstyle=\color{mygreen}, % Adjust the color here
    stringstyle=\color{orange},
    numbers=left,
    numberstyle=\tiny,
    stepnumber=1,
    numbersep=5pt,
    backgroundcolor=\color{gray!5},
    frame=single,
    breaklines=true,
    breakatwhitespace=true,
    captionpos=b,
    tabsize=4,
    showspaces=false,
    showstringspaces=false
}

\lstset{style=mystyle}


\usepackage{amsmath}
\rfoot{\thepage}
\usepackage{afterpage}
\usepackage{graphicx}


%-----------------------
% Begin document
%-----------------------
\begin{document}

\section{Problem Definition}

The Miller-Rabin Test is a test to determine the probability that a number is a composite number. Repeated implementations can prove to a sufficient degree whether a number is prime or composite. With each test, the completion and determination that variable a is a Miller-Rabin Witness is 75\%, meaning that with $i$ iterations, the odds that the tested number is prime is $0.25^{i}$. With a large i, this is one of the most effective tests to determine very large prime numbers. 

\section{Code Workthrough}

I programmed the Miller-Rabin Test in Python, only using the $random$ and $math$ libraries. 

\begin{lstlisting}
import random
import math
\end{lstlisting}

Random features functions that generate a random number in a specific range (among others). I use this in the code to find different $a$ values that may be used as potential witnesses to the compositeness of N. Math is used to perform the gcd function, which I need to determine if the Miller-Rabin test makes sense for a certain $a$ value. 

After this, I created a function to run the Miller-Rabin Test. This function takes in two variables:

\begin{enumerate}
    \item N, the number with which I am testing the primality.
    \item i, the number of iterations. Since the odds that a tested number is prime is $0.25^{i}$, I found it computationally reasonable to set i equal to anywhere from 10-100, which offers 
\end{enumerate}
    
Starting the code as follows: 

\vspace{3mm}

\begin{lstlisting}
def miller_rabin_test(N, i=10):
    """
    Perform the Miller-Rabin primality test on the number N.
    returns true if the test is inconclusive/fails and N could be prime, and false otherwise if it is composite
    """
    if N == 2 or N == 3:
        return True
    if N == 1 or N % 2 == 0:
        return False
\end{lstlisting}

The above section of the program ensures that when N is less than 4, there are no errors. Likewise, it ends the test if the number being tested is even (N \% 2), since by definition it must be composite. 

\vspace{2mm}

\begin{lstlisting}
    #Writing N as 2^k*q
    k = 0
    q = N - 1
    while q % 2 == 0:
        k += 1
        q //= 2
\end{lstlisting}

This while loop puts the $N-1$ value which I am testing into the form of $2^{k}q$. 

\vspace{2mm}

\begin{lstlisting}
    #witness loop ( repeating k times unless broken by the return )
    for j in range(k):
        #choosing a random int between 2 and N-1-1 or N-2
        a = random.randint(2, N - 2)
        #breaks the loop if the gcd ever is not 1 (N is divisible by a or a product of it, automatically making it not prime)
        if math.gcd(a, N) != 1:
            return False

        #defines x as a to the q mod N
        x = pow(a, q, N)

        #if x is ever 1 or -1, it will skip to the next iteration of the for loop above (k+1)
        if x == 1 or x == N - 1:
            continue
\end{lstlisting}

The expression x = pow(a, k, N) calculates the modular exponentiation of $a$ raised to the power of $k$, modulo N. Thus, as the problem is defined:

\[x = a^{k}modN\]

To finish off the above "witness loop," I added the steps for which the loop iterates from i = 1 to k-1, and if at any point a is equal to -1modN, the test fails, making it more likely to be prime. If this scenario happens, it breaks out of the for loop and returns. 

\clearpage

\begin{lstlisting}
    for j in range(k - 1):
            x = pow(x, 2, N)

            #searching for when x = -1modN
            if x == N - 1:
                break
        else:
            return False
    
    #only returns true if all of the iterations are also true
    return True
\end{lstlisting}

\clearpage

\section{Example}

In this example, I went online to find large prime numbers and determined that 222334565193649 is prime. Plugging this into the code, as below, it is returned as probably prime (100 iterations). 

\begin{lstlisting}
#example
primality_test = 222334565193647  
iterations = 100       

result = miller_rabin_test(primality_test, iterations)
print(primality_test, "is", ('probably prime' if result else 'composite'))
\end{lstlisting}

In contrast, when plugging in another number, for example, 222334565193647 (the number above minus two), the code determines that the result is probably a composite. 

\begin{lstlisting}
#example
primality_test = 222334565193649 
iterations = 100       

result = miller_rabin_test(primality_test, iterations)
print(primality_test, "is", ('probably prime' if result else 'composite'))
\end{lstlisting}

In an additional proof of the legitimacy of the algorithm, I ran the algorithm for 100 iterations for each number n from 1 to 1000. This printed all 168 primes less than 1000, as noted in the textbook.  

\begin{lstlisting}
primes_list = []
primes_count = 0
for i in range(1000):
    result = miller_rabin_test(i, 100)
    if(result):
        primes_list.append(i)
        primes_count = primes_count + 1
print(primes_list)
print(primes_count)
\end{lstlisting}

\clearpage

\section{Code Appendix}

\begin{lstlisting}
import random
import math

def miller_rabin_test(N, i=10):
    """
    Perform the Miller-Rabin primality test on the number N.
    returns true if the test is inconclusive/fails and N could be prime, and false otherwise if it is composite
    """
    if N == 2 or N == 3:
        return True
    if N == 1 or N % 2 == 0:
        return False

    #writing N as 2^k*q
    k = 0
    q = N - 1
    while q % 2 == 0:
        k += 1
        q //= 2
        
    #witness loop ( repeating k times unless broken by the return )
    for j in range(k):
        #choosing a random int between 2 and N-1-1 or N-2
        a = random.randint(2, N - 2)
        #breaks the loop if the gcd ever is not 1 (N is divisible by a or a product of it, automatically making it not prime)
        if math.gcd(a, N) != 1:
            return False

        #defines x as a to the q mod N
        x = pow(a, q, N)

        #if x is ever 1 or -1, it will skip to the next iteration of the for loop above (k+1)
        if x == 1 or x == N - 1:
            continue

        for j in range(k - 1):
            x = pow(x, 2, N)

            #searching for when x = -1modN
            if x == N - 1:
                break
        else:
            return False
    
    #only returns true if all of the iterations are also true
    return True

#example
primality_test = 222334565193647  
iterations = 100       

result = miller_rabin_test(primality_test, iterations)
print(primality_test, "is", ('probably prime' if result else 'composite'))

#additional example
primes_list = []
primes_count = 0
for i in range(1000):
    result = miller_rabin_test(i, 100)
    if(result):
        primes_list.append(i)
        primes_count = primes_count + 1
print(primes_list)
print(primes_count)
\end{lstlisting}

\end{document}

